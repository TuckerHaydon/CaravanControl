%
%%%%%%%%%%%%%%%%%%%%%%%%%%%%%%%%%%%%%%%%%%%%%%%%%%%%%%%%%%%%%%%%%%%%%%%%%%

\documentclass[12pt,onecolumn,reqno]{amsart}
\usepackage{subcaption,wrapfig,graphicx,booktabs,fancyhdr,amsmath,amsfonts}
\usepackage{cite,bm,amssymb,amsthm,wasysym,url,multirow,float}
\newcommand{\vb}{\boldsymbol}
\newcommand{\vbh}[1]{\hat{\boldsymbol{#1}}}
\newcommand{\vbb}[1]{\bar{\boldsymbol{#1}}}
\newcommand{\vbt}[1]{\tilde{\boldsymbol{#1}}}
\newcommand{\vbs}[1]{{\boldsymbol{#1}}^*}
\newcommand{\vbd}[1]{\dot{{\boldsymbol{#1}}}}
\newcommand{\by}{\times}
\newcommand{\tr}{{\rm tr}}
\newcommand{\sfrac}[2]{\textstyle \frac{#1}{#2}}
\newcommand{\ba}{\begin{array}}
\newcommand{\ea}{\end{array}}
\renewcommand{\earth}{\oplus}
\newcommand{\sinc}{{\rm sinc}}
\renewcommand{\equiv}{\triangleq}
\newcommand{\cnr}{C/N_0}
\newcommand{\sgn}{\rm sgn}
\renewcommand{\Re}{\mathbb{R}}
\renewcommand{\Im}{\mathbb{I}}
\newcommand{\E}[1]{\mathbb{E}\left[ #1 \right]}
\newcommand{\norm}[1]{\left\lVert#1\right\rVert}
\DeclareMathAlphabet{\mathpzc}{OT1}{pzc}{m}{it}

\topmargin = 0 mm 
\oddsidemargin = -1 mm 
\evensidemargin = -1 mm
\headheight = 0 mm 
\headsep = 8 mm 
\textheight = 220 mm 
\textwidth =170 mm 
\parindent = 0 mm
\parskip = 4 mm 

\setcounter{MaxMatrixCols}{15}

%%% ----------------------------------------------------------------------
\begin{document}
\title[]{Feedback Control Systems Final Report \\ Caravan Control}
\author[]{Tucker Haydon and Connor Brashar}
\address{The University of Texas at Austin}
\email{thaydon@utexas.edu, connor.brashar@utexas.edu}
\date{\today}
\begin{abstract}
  This is the project abstract.
\end{abstract}
\maketitle


%%% ----------------------------------------------------------------------
\section{Scenario}
You are an engineer in charge of designing and implementing an estimation and
control system for a fleet of self-driving semi-trucks. Due to government
regulation, the truck fleet can only operate in the self-driving mode when they
are on long, straight stretches of highway between cities. In order to reduce
costs and conserve fuel, the trucks drive very closely together at a
pre-determined speed in `caravan formation'. By driving very closely together,
the trucks can draft off of one another, reduce drag, and save fuel by upwards
of 21\% \cite{bonnet2000fuel}.

For the system you are designing, only three trucks will be in the caravan. The
caravan is equipped with the following sensor suite: 
\begin{enumerate}
  \item The lead truck is equipped with a GPS receiver that measures its
    position at 1 Hz.
  \item The two following trucks are equipped with range sensors that measure
    the relative position between themselves and the truck in front of them at
    10 Hz.
  \item All three trucks are equipped with an IMU that measures respective
    acceleration at 100 Hz.
\end{enumerate}

Each truck may be independently controlled and may be instructed to either
accelerate or decelerate.

%%% ----------------------------------------------------------------------
\section{Nominal System Description}
\label{sec:nominal_system}

Define the system state.
\begin{equation}
  \vec{x} = 
  \begin{bmatrix}
    x_{1} \\
    x_{2} \\
    x_{3} \\
    v_{1} \\
    v_{2} \\
    v_{3}
  \end{bmatrix}
\end{equation}

Define the system input.
\begin{equation}
  \vec{u} = 
  \begin{bmatrix}
    a_{1} \\
    a_{2} \\
    a_{3}
  \end{bmatrix}
\end{equation}

Define the system dynamics.
\begin{equation}
  \dot{\vec{x}}
  =
  \begin{bmatrix}
    v_{1} \\
    v_{2} \\
    v_{3} \\
    a_{1} \\
    a_{2} \\
    a_{3}
  \end{bmatrix}
  =
  \underbrace{
  \begin{bmatrix}
    0 & 0 & 0 & 1 & 0 & 0  \\
    0 & 0 & 0 & 0 & 1 & 0  \\
    0 & 0 & 0 & 0 & 0 & 1  \\
    0 & 0 & 0 & 0 & 0 & 0  \\
    0 & 0 & 0 & 0 & 0 & 0  \\
    0 & 0 & 0 & 0 & 0 & 0  \\
  \end{bmatrix}
  }_{A}
  \vec{x}
  +
  \underbrace{
  \begin{bmatrix}
    0 & 0 & 0 \\
    0 & 0 & 0 \\
    0 & 0 & 0 \\
    1 & 0 & 0 \\
    0 & 1 & 0 \\
    0 & 0 & 1 \\
  \end{bmatrix}
  }_{B}
  \vec{u}
\end{equation}

Define the system observer.
\begin{equation}
  \vec{y} = 
  \begin{bmatrix}
    x_{1}             \\
    \Delta x_{12}     \\
    \Delta x_{23}     \\
    a_{1}             \\
    a_{2}             \\
    a_{3}             
  \end{bmatrix}
  =
  \underbrace{
  \begin{bmatrix}
    1 & 0 & 0 & 0 & 0 & 0   \\
    1 & -1 & 0 & 0 & 0 & 0  \\
    0 & 1 & -1 & 0 & 0 & 0  \\
    0 & 0 & 0 & 0 & 0 & 0   \\
    0 & 0 & 0 & 0 & 0 & 0   \\
    0 & 0 & 0 & 0 & 0 & 0  
  \end{bmatrix}
  }_{C}
  \vec{x}
  +
  \underbrace{
  \begin{bmatrix}
    0 & 0 & 0 \\
    0 & 0 & 0 \\
    0 & 0 & 0 \\
    1 & 0 & 0 \\
    0 & 1 & 0 \\
    0 & 0 & 1
  \end{bmatrix}
  }_{D}
  \vec{u}
\end{equation}

\subsection{Observability}
Using the nominal system, determine whether or not the system is observable. The
nominal system is linear and time-invariant, so the observability Grammian is
simplified:

\begin{equation}
  W_{o} = 
  \begin{bmatrix}
    C      \\
    CA     \\
    CA^2   \\
    \vdots \\
    CA^{n-1}
  \end{bmatrix}
\end{equation}

If the rank of $W_{o} = n$ then the system is observable. \textbf{The nominal
system is observable}.


\subsection{Controllability}
Using the nominal system, determine whether or not the system is controllable. The
nominal system is linear and time-invariant, so the controllability Grammian is
simplified:

\begin{equation}
  W_{c} = 
  \begin{bmatrix}
    B & AB & A^2B & \hdots & A^{n-1}N
  \end{bmatrix}
\end{equation}

If the rank of $W_{c} = n$ then the system is controllable. \textbf{The nominal
system is controllable}.

\subsection{Conclusion}
The fact that the nominal system is both observable and controllable indicates
that it is well-designed and amenable to both estimation and control.


%%% ----------------------------------------------------------------------
\section{Estimator}
TODO Connor

%%% ----------------------------------------------------------------------
\section{Control System}
Controllers drive systems to the origin. The nominal system, however, should not
be driven to the origin. Instead an error-state system should be specified where
the error is the deviation of the nominal system from a reference signal.

First, define the reference signal. From the problem statement, the goal is to
maintain a specified distance between the trucks and to keep that at a constant
velocity.
\begin{equation}
  \vec{x_{r}} = 
  \begin{bmatrix}
    x_{1_{r}}         \\
    \Delta x_{12_{r}} \\
    \Delta x_{23_{r}} \\
    v_{1_{r}}         \\
    v_{2_{r}}         \\
    v_{3_{r}}
  \end{bmatrix}
\end{equation}

The error signal is the difference between the nominal and the reference state.
\begin{equation}
  \vec{x_{e}} = 
  \begin{bmatrix}
    x_{1} - x_{1_{r}}                 \\ 
    x_{1} - x_{2} - \Delta x_{12_{r}} \\
    x_{2} - x_{3} - \Delta x_{23_{r}} \\
    v_{1} - v_{1_{r}}                 \\
    v_{2} - v_{2_{r}}                 \\
    v_{3} - v_{3_{r}}
  \end{bmatrix}
\end{equation}

The error signal is an affine transform using both the nominal state and the
reference signal. The state dynamics for an affine transform don't conform to
the standard $\dot{x} = Ax + Bu$ form. However, the standard form can be
composed via the following trick: append the reference signal to the nominal
state and define zero dynamics and full observability for the reference
substate.

Define the error state and stack it on top of the reference signal. 
\begin{equation}
  \vec{e} = 
  \begin{bmatrix}
    x_{1}             \\
    x_{2}             \\
    x_{3}             \\
    v_{1}             \\
    v_{2}             \\
    v_{3}             \\ 
    \Delta x_{12_{r}} \\
    \Delta x_{23_{r}} \\
    v_{1_{r}}         \\
  \end{bmatrix}
\end{equation}

Define the error state dynamics.
\begin{equation}
  \dot{\vec{e}} = 
  \begin{bmatrix}
    v_{1}             \\
    \Delta v_{12}     \\
    \Delta v_{23}     \\
    a_{1}             \\
    \Delta a_{12}     \\
    \Delta a_{23}     \\ 
    0                 \\
    0                 \\
    0                 \\
    0                 \\
    0                
  \end{bmatrix}
  =
  \underbrace{
  \begin{bmatrix}
    0 & 0 & 0 & 1 & 0 & 0 & 0 & 0 & 0 & 0 & 0 \\
    0 & 0 & 0 & 0 & 1 & 0 & 0 & 0 & 0 & 0 & 0 \\
    0 & 0 & 0 & 0 & 0 & 1 & 0 & 0 & 0 & 0 & 0 \\
    0 & 0 & 0 & 0 & 0 & 0 & 0 & 0 & 0 & 0 & 0 \\
    0 & 0 & 0 & 0 & 0 & 0 & 0 & 0 & 0 & 0 & 0 \\
    0 & 0 & 0 & 0 & 0 & 0 & 0 & 0 & 0 & 0 & 0 \\
    0 & 0 & 0 & 0 & 0 & 0 & 0 & 0 & 0 & 0 & 0 \\
    0 & 0 & 0 & 0 & 0 & 0 & 0 & 0 & 0 & 0 & 0 \\
    0 & 0 & 0 & 0 & 0 & 0 & 0 & 0 & 0 & 0 & 0 \\
    0 & 0 & 0 & 0 & 0 & 0 & 0 & 0 & 0 & 0 & 0 \\
    0 & 0 & 0 & 0 & 0 & 0 & 0 & 0 & 0 & 0 & 0
  \end{bmatrix}
  }_{A}
  \vec{e}
  +
  \underbrace{
  \begin{bmatrix}
    0 & 0 & 0 \\
    0 & 0 & 0 \\
    0 & 0 & 0 \\
    1 & 0 & 0 \\
    1 & -1 & 0 \\
    0 & 1 & -1 \\
    0 & 0 & 0 \\
    0 & 0 & 0 \\
    0 & 0 & 0 \\
    0 & 0 & 0 \\
    0 & 0 & 0
  \end{bmatrix}
  }_{B}
  \vec{u}
\end{equation}

Define the error state observer equations.
\begin{equation}
  \vec{y} = 
  \begin{bmatrix}
    x_{1}             \\
    \Delta x_{12}     \\
    \Delta x_{23}     \\
    a_{1}             \\
    a_{2}             \\
    a_{3}             \\
    \Delta x_{12_{r}} \\
    \Delta x_{23_{r}} \\
    v_{1_{r}}         \\
    \Delta v_{12_{r}} \\
    \Delta v_{23_{r}}
  \end{bmatrix}
  =
  \underbrace{
  \begin{bmatrix}
    1 & 0 & 0 & 0 & 0 & 0 & 0 & 0 & 0 & 0 & 0 \\
    0 & 1 & 0 & 0 & 0 & 0 & 0 & 0 & 0 & 0 & 0 \\
    0 & 0 & 1 & 0 & 0 & 0 & 0 & 0 & 0 & 0 & 0 \\
    0 & 0 & 0 & 0 & 0 & 0 & 0 & 0 & 0 & 0 & 0 \\
    0 & 0 & 0 & 0 & 0 & 0 & 0 & 0 & 0 & 0 & 0 \\
    0 & 0 & 0 & 0 & 0 & 0 & 0 & 0 & 0 & 0 & 0 \\
    0 & 0 & 0 & 0 & 0 & 0 & 1 & 0 & 0 & 0 & 0 \\
    0 & 0 & 0 & 0 & 0 & 0 & 0 & 1 & 0 & 0 & 0 \\
    0 & 0 & 0 & 0 & 0 & 0 & 0 & 0 & 1 & 0 & 0 \\
    0 & 0 & 0 & 0 & 0 & 0 & 0 & 0 & 0 & 1 & 0 \\
    0 & 0 & 0 & 0 & 0 & 0 & 0 & 0 & 0 & 0 & 1
  \end{bmatrix}
  }_{C}
  \vec{e}
  +
  \underbrace{
  \begin{bmatrix}
    0 & 0 & 0 \\
    0 & 0 & 0 \\
    0 & 0 & 0 \\
    1 & 0 & 0 \\
    0 & 1 & 0 \\
    0 & 0 & 1 \\
    0 & 0 & 0 \\
    0 & 0 & 0 \\
    0 & 0 & 0 \\
    0 & 0 & 0 \\
    0 & 0 & 0 
  \end{bmatrix}
  }_{D}
  \vec{u}
\end{equation}
%%% ----------------------------------------------------------------------
\section{Simulation}


%%% ----------------------------------------------------------------------
\section{Results}


%%% ----------------------------------------------------------------------
\bibliographystyle{ieeetr}
\bibliography{report.bib}
\end{document}


