%
%%%%%%%%%%%%%%%%%%%%%%%%%%%%%%%%%%%%%%%%%%%%%%%%%%%%%%%%%%%%%%%%%%%%%%%%%%

\documentclass[12pt,onecolumn,reqno]{amsart}
\usepackage{subcaption,wrapfig,graphicx,booktabs,fancyhdr,amsmath,amsfonts}
\usepackage{cite,bm,amssymb,amsthm,wasysym,url,multirow,float}
\newcommand{\vb}{\boldsymbol}
\newcommand{\vbh}[1]{\hat{\boldsymbol{#1}}}
\newcommand{\vbb}[1]{\bar{\boldsymbol{#1}}}
\newcommand{\vbt}[1]{\tilde{\boldsymbol{#1}}}
\newcommand{\vbs}[1]{{\boldsymbol{#1}}^*}
\newcommand{\vbd}[1]{\dot{{\boldsymbol{#1}}}}
\newcommand{\by}{\times}
\newcommand{\tr}{{\rm tr}}
\newcommand{\sfrac}[2]{\textstyle \frac{#1}{#2}}
\newcommand{\ba}{\begin{array}}
\newcommand{\ea}{\end{array}}
\renewcommand{\earth}{\oplus}
\newcommand{\sinc}{{\rm sinc}}
\renewcommand{\equiv}{\triangleq}
\newcommand{\cnr}{C/N_0}
\newcommand{\sgn}{\rm sgn}
\renewcommand{\Re}{\mathbb{R}}
\renewcommand{\Im}{\mathbb{I}}
\newcommand{\E}[1]{\mathbb{E}\left[ #1 \right]}
\newcommand{\norm}[1]{\left\lVert#1\right\rVert}
\DeclareMathAlphabet{\mathpzc}{OT1}{pzc}{m}{it}

\topmargin = 0 mm 
\oddsidemargin = -1 mm 
\evensidemargin = -1 mm
\headheight = 0 mm 
\headsep = 8 mm 
\textheight = 220 mm 
\textwidth =170 mm 
\parindent = 0 mm
\parskip = 4 mm 

\setcounter{MaxMatrixCols}{15}

%%% ----------------------------------------------------------------------
\begin{document}
\title[]{Feedback Control Systems Final Report \\ Caravan Control}
\author[]{Tucker Haydon and Connor Brashar}
\address{The University of Texas at Austin}
\email{thaydon@utexas.edu, connor.brashar@utexas.edu}
\date{\today}
\begin{abstract}
  This is the project abstract.
\end{abstract}
\maketitle


%%% ----------------------------------------------------------------------
\section{Scenario}
You are an engineer in charge of designing and implementing an estimation and
control system for a fleet of self-driving semi-trucks. Due to government
regulation, the truck fleet can only operate in the self-driving mode when they
are on long, straight stretches of highway between cities. In order to reduce
costs and conserve fuel, the trucks drive very closely together at a
pre-determined speed in `caravan formation'. By driving very closely together,
the trucks can draft off of one another, reduce drag, and save fuel by upwards
of 21\% \cite{bonnet2000fuel}.

For the system you are designing, only three trucks will be in the caravan. The
caravan is equipped with the following sensor suite: 
\begin{enumerate}
  \item The lead truck is equipped with a GPS receiver that measures its
    position at 1 Hz.
  \item The two following trucks are equipped with range sensors that measure
    the relative position between themselves and the truck in front of them at
    10 Hz.
  \item All three trucks are equipped with an IMU that measures respective
    acceleration at 100 Hz.
\end{enumerate}

Each truck may be independently controlled and may be instructed to either
accelerate or decelerate.

%%% ----------------------------------------------------------------------
\section{System Description}
Define the system state.
\begin{equation}
  \vec{x} = 
  \begin{bmatrix}
    x_{1} \\
    x_{2} \\
    x_{3} \\
    v_{1} \\
    v_{2} \\
    v_{3}
  \end{bmatrix}
\end{equation}

Define the reference states.
\begin{equation}
  \vec{x_{r}} = 
  \begin{bmatrix}
    \Delta x_{12_{r}} \\
    \Delta x_{23_{r}} \\
    v_{1_{r}}         \\
    \Delta v_{12_{r}} \\
    \Delta v_{23_{r}}
  \end{bmatrix}
\end{equation}

Define the input vector.
\begin{equation}
  \vec{u} = 
  \begin{bmatrix}
    a_{1} \\
    a_{2} \\
    a_{3}
  \end{bmatrix}
\end{equation}

Define the error state and stack it on top of the reference signal.
\begin{equation}
  \vec{e} = 
  \begin{bmatrix}
    x_{1}             \\
    \Delta x_{12}     \\
    \Delta x_{23}     \\
    v_{1}             \\
    \Delta v_{12}     \\
    \Delta v_{23}     \\ 
    \Delta x_{12_{r}} \\
    \Delta x_{23_{r}} \\
    v_{1_{r}}         \\
    \Delta v_{12_{r}} \\
    \Delta v_{23_{r}}
  \end{bmatrix}
\end{equation}

Define the error state dynamics.
\begin{equation}
  \dot{\vec{e}} = 
  \begin{bmatrix}
    v_{1}             \\
    \Delta v_{12}     \\
    \Delta v_{23}     \\
    a_{1}             \\
    \Delta a_{12}     \\
    \Delta a_{23}     \\ 
    0                 \\
    0                 \\
    0                 \\
    0                 \\
    0                
  \end{bmatrix}
  =
  \underbrace{
  \begin{bmatrix}
    0 & 0 & 0 & 1 & 0 & 0 & 0 & 0 & 0 & 0 & 0 \\
    0 & 0 & 0 & 0 & 1 & 0 & 0 & 0 & 0 & 0 & 0 \\
    0 & 0 & 0 & 0 & 0 & 1 & 0 & 0 & 0 & 0 & 0 \\
    0 & 0 & 0 & 0 & 0 & 0 & 0 & 0 & 0 & 0 & 0 \\
    0 & 0 & 0 & 0 & 0 & 0 & 0 & 0 & 0 & 0 & 0 \\
    0 & 0 & 0 & 0 & 0 & 0 & 0 & 0 & 0 & 0 & 0 \\
    0 & 0 & 0 & 0 & 0 & 0 & 0 & 0 & 0 & 0 & 0 \\
    0 & 0 & 0 & 0 & 0 & 0 & 0 & 0 & 0 & 0 & 0 \\
    0 & 0 & 0 & 0 & 0 & 0 & 0 & 0 & 0 & 0 & 0 \\
    0 & 0 & 0 & 0 & 0 & 0 & 0 & 0 & 0 & 0 & 0 \\
    0 & 0 & 0 & 0 & 0 & 0 & 0 & 0 & 0 & 0 & 0
  \end{bmatrix}
  }_{A}
  \vec{e}
  +
  \underbrace{
  \begin{bmatrix}
    0 & 0 & 0 \\
    0 & 0 & 0 \\
    0 & 0 & 0 \\
    1 & 0 & 0 \\
    1 & -1 & 0 \\
    0 & 1 & -1 \\
    0 & 0 & 0 \\
    0 & 0 & 0 \\
    0 & 0 & 0 \\
    0 & 0 & 0 \\
    0 & 0 & 0
  \end{bmatrix}
  }_{B}
  \vec{u}
\end{equation}

Define the error state observer equations.
\begin{equation}
  \vec{y} = 
  \begin{bmatrix}
    x_{1}             \\
    \Delta x_{12}     \\
    \Delta x_{23}     \\
    a_{1}             \\
    a_{2}             \\
    a_{3}             \\
    \Delta x_{12_{r}} \\
    \Delta x_{23_{r}} \\
    v_{1_{r}}         \\
    \Delta v_{12_{r}} \\
    \Delta v_{23_{r}}
  \end{bmatrix}
  =
  \underbrace{
  \begin{bmatrix}
    1 & 0 & 0 & 0 & 0 & 0 & 0 & 0 & 0 & 0 & 0 \\
    0 & 1 & 0 & 0 & 0 & 0 & 0 & 0 & 0 & 0 & 0 \\
    0 & 0 & 1 & 0 & 0 & 0 & 0 & 0 & 0 & 0 & 0 \\
    0 & 0 & 0 & 0 & 0 & 0 & 0 & 0 & 0 & 0 & 0 \\
    0 & 0 & 0 & 0 & 0 & 0 & 0 & 0 & 0 & 0 & 0 \\
    0 & 0 & 0 & 0 & 0 & 0 & 0 & 0 & 0 & 0 & 0 \\
    0 & 0 & 0 & 0 & 0 & 0 & 1 & 0 & 0 & 0 & 0 \\
    0 & 0 & 0 & 0 & 0 & 0 & 0 & 1 & 0 & 0 & 0 \\
    0 & 0 & 0 & 0 & 0 & 0 & 0 & 0 & 1 & 0 & 0 \\
    0 & 0 & 0 & 0 & 0 & 0 & 0 & 0 & 0 & 1 & 0 \\
    0 & 0 & 0 & 0 & 0 & 0 & 0 & 0 & 0 & 0 & 1
  \end{bmatrix}
  }_{C}
  \vec{e}
  +
  \underbrace{
  \begin{bmatrix}
    0 & 0 & 0 \\
    0 & 0 & 0 \\
    0 & 0 & 0 \\
    1 & 0 & 0 \\
    0 & 1 & 0 \\
    0 & 0 & 1 \\
    0 & 0 & 0 \\
    0 & 0 & 0 \\
    0 & 0 & 0 \\
    0 & 0 & 0 \\
    0 & 0 & 0 
  \end{bmatrix}
  }_{D}
  \vec{u}
\end{equation}


%%% ----------------------------------------------------------------------
\section{Analysis}


%%% ----------------------------------------------------------------------
\section{Simulation}


%%% ----------------------------------------------------------------------
\section{Results}


%%% ----------------------------------------------------------------------
\bibliographystyle{ieeetr}
\bibliography{report.bib}
\end{document}


