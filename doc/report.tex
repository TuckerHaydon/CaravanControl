%
%%%%%%%%%%%%%%%%%%%%%%%%%%%%%%%%%%%%%%%%%%%%%%%%%%%%%%%%%%%%%%%%%%%%%%%%%%

\documentclass[12pt,onecolumn,reqno]{amsart}
\usepackage{subcaption,wrapfig,graphicx,booktabs,fancyhdr,amsmath,amsfonts}
\usepackage{cite,bm,amssymb,amsthm,wasysym,url,multirow,float,mathtools,hyperref}
\usepackage{float}
\newcommand{\vb}{\boldsymbol}
\newcommand{\vbh}[1]{\hat{\boldsymbol{#1}}}
\newcommand{\vbb}[1]{\bar{\boldsymbol{#1}}}
\newcommand{\vbt}[1]{\tilde{\boldsymbol{#1}}}
\newcommand{\vbs}[1]{{\boldsymbol{#1}}^*}
\newcommand{\vbd}[1]{\dot{{\boldsymbol{#1}}}}
\newcommand{\by}{\times}
\newcommand{\tr}{{\rm tr}}
\newcommand{\sfrac}[2]{\textstyle \frac{#1}{#2}}
\newcommand{\ba}{\begin{array}}
\newcommand{\ea}{\end{array}}
\renewcommand{\earth}{\oplus}
\newcommand{\sinc}{{\rm sinc}}
\renewcommand{\equiv}{\triangleq}
\newcommand{\cnr}{C/N_0}
\newcommand{\sgn}{\rm sgn}
\renewcommand{\Re}{\mathbb{R}}
\renewcommand{\Im}{\mathbb{I}}
\newcommand{\E}[1]{\mathbb{E}\left[ #1 \right]}
\newcommand{\norm}[1]{\left\lVert#1\right\rVert}
\DeclareMathAlphabet{\mathpzc}{OT1}{pzc}{m}{it}

\topmargin = 0 mm 
\oddsidemargin = -1 mm 
\evensidemargin = -1 mm
\headheight = 0 mm 
\headsep = 8 mm 
\textheight = 220 mm 
\textwidth =170 mm 
\parindent = 0 mm
\parskip = 4 mm 

\setcounter{MaxMatrixCols}{15}

%%% ----------------------------------------------------------------------
\begin{document}
\title[]{Feedback Control Systems Final Report \\ Caravan Control}
\author[]{Tucker Haydon and Connor Brashar}
\address{The University of Texas at Austin}
\email{thaydon@utexas.edu, connor.brashar@utexas.edu}
\date{\today}
\begin{abstract}
  A fleet of self-driving trucks is simulated on a long, flat, straight piece of
  road. The fleet is brought into a `caravan' formation within ten minutes via a
  discrete-time, finite horizon LQG controller. The trucks have limited sensing
  capabilities: GPS for the lead truck and range sensors between all following
  trucks. Finally, a hands-on museum exhibit mirroring the simulation and
  demonstrating autonomy and control concepts is proposed.
\end{abstract}
\maketitle


%%% ----------------------------------------------------------------------
\section{Scenario}
You are an engineer in charge of designing and implementing an estimation and
control system for a fleet of self-driving semi-trucks. Due to government
regulation, the truck fleet can only operate in the self-driving mode when they
are on long, straight stretches of highway between cities. In order to reduce
costs and conserve fuel, the trucks drive very closely together at a
pre-determined speed in `caravan formation'. By driving very closely together,
the trucks can draft off of one another, reduce drag, and save fuel by upwards
of 21\% \cite{bonnet2000fuel}.

For the system you are designing, only three trucks will be in the caravan. The
caravan is equipped with the following sensor suite: 
\begin{enumerate}
  \item The lead truck is equipped with a GPS receiver that measures its
    position at 1 Hz.
  \item The two following trucks are equipped with range sensors that measure
    the relative position between themselves and the truck in front of them at
    10 Hz.
\end{enumerate}

Each truck may be independently controlled and may be instructed to either
accelerate or decelerate.

%%% ----------------------------------------------------------------------
\section{Nominal System Description}
\label{sec:nominal_system}

Define the system state.
\begin{equation}
  \vec{x} = 
  \begin{bmatrix}
    x_{1} \\
    x_{2} \\
    x_{3} \\
    v_{1} \\
    v_{2} \\
    v_{3}
  \end{bmatrix}
\end{equation}

Define the system input.
\begin{equation}
  \vec{u} = 
  \begin{bmatrix}
    a_{1} \\
    a_{2} \\
    a_{3}
  \end{bmatrix}
\end{equation}

Define the system dynamics.
\begin{equation}
  \dot{\vec{x}}
  =
  \begin{bmatrix}
    v_{1} \\
    v_{2} \\
    v_{3} \\
    a_{1} \\
    a_{2} \\
    a_{3}
  \end{bmatrix}
  =
  \underbrace{
  \begin{bmatrix}
    0 & 0 & 0 & 1 & 0 & 0  \\
    0 & 0 & 0 & 0 & 1 & 0  \\
    0 & 0 & 0 & 0 & 0 & 1  \\
    0 & 0 & 0 & 0 & 0 & 0  \\
    0 & 0 & 0 & 0 & 0 & 0  \\
    0 & 0 & 0 & 0 & 0 & 0  \\
  \end{bmatrix}
  }_{A}
  \vec{x}
  +
  \underbrace{
  \begin{bmatrix}
    0 & 0 & 0 \\
    0 & 0 & 0 \\
    0 & 0 & 0 \\
    1 & 0 & 0 \\
    0 & 1 & 0 \\
    0 & 0 & 1 \\
  \end{bmatrix}
  }_{B}
  \vec{u}
\end{equation}

Define the system observer equation.
\begin{equation}
  \vec{y} = 
  \begin{bmatrix}
    x_{1}             \\
    \Delta x_{12}     \\
    \Delta x_{23}     \\
  \end{bmatrix}
  =
  \underbrace{
  \begin{bmatrix}
    1 & 0 & 0 & 0 & 0 & 0   \\
    1 & -1 & 0 & 0 & 0 & 0  \\
    0 & 1 & -1 & 0 & 0 & 0  \\
    0 & 0 & 0 & 0 & 0 & 0   \\
    0 & 0 & 0 & 0 & 0 & 0   \\
    0 & 0 & 0 & 0 & 0 & 0  
  \end{bmatrix}
  }_{C}
  \vec{x}
\end{equation}

\subsection{Observability}
Using the nominal system, determine whether or not the system is observable. The
nominal system is linear and time-invariant, so the observability Grammian is
simplified:

\begin{equation}
  W_{o} = 
  \begin{bmatrix}
    C      \\
    CA     \\
    CA^2   \\
    \vdots \\
    CA^{n-1}
  \end{bmatrix}
\end{equation}

If the rank of $W_{o} = n$ then the system is observable. \textbf{The nominal
system is observable}.


\subsection{Controllability}
Using the nominal system, determine whether or not the system is controllable. The
nominal system is linear and time-invariant, so the controllability Grammian is
simplified:

\begin{equation}
  W_{c} = 
  \begin{bmatrix}
    B & AB & A^2B & \hdots & A^{n-1}N
  \end{bmatrix}
\end{equation}

If the rank of $W_{c} = n$ then the system is controllable. \textbf{The nominal
system is controllable}.

\subsection{Conclusion}
The fact that the nominal system is both observable and controllable indicates
that it is well-designed and amenable to both estimation and control.

\subsection{Discrete-Time Dynamics} \label{sec:discrete_time}
Software implementations of continuous kinematic systems must first discretize
the system dynamics. Many sensors and computers cannot produce nor consume
continuous signals. As a result, system dynamics or measurements are
represented as discrete systems. 

Define the discrete system. 
\begin{align*}
  x_{k+1} = A_{d} x_{k} + B_{d} u_k \\
  y_{k} = C x_{k} + D u_{k}
\end{align*}

For a discretized system, only the system dynamics matrices (A, B) change:
\begin{align*}
  A_{d} = e^{A \cdot \Delta T} \\
  B_d = \int_{0}^{\Delta T} e^{A \lambda} d \lambda B
\end{align*}
where $\Delta T$ is the discrete time interval. For this project, $\Delta T$ was
chosen to match the smallest measurement interval, $\Delta T = 0.1$.

These discrete-time dynamics are used in both the estimator and the controller.


%%% ----------------------------------------------------------------------
\section{Estimator} \label{sec:Kalman_filter}
Estimators are designed to predict each state in a system, if that state
information is not directly observable. Often, measurements are taken that
contain indirect information about a state. For example, accelerometer
measurements must be integrated twice to get a position measurement. Naturally,
there is some noise and inaccuracy in this process, but provided the system is
fully observable, a Kalman filter can be implemented on a linear system to get
direct estimates of full state variables from measurement data. 

Kalman filters are a simple form of estimator that works on linear systems of the form:

\begin{align*}
    \dot{x} &= A x + B u + v \\
    z &= Hx + w
\end{align*}

where $v$ and $w$ are respectively random process and measurement noise, and $H
= C$. We note that the Kalman filter assumes no user input related to the
measurement.  Because user input is known and prescribed by the controller, we
don't wish to include it in our states that we are estimating. As a result, our
Kalman Filter will not try to estimate acceleration, only position and velocity.
The Kalman filter model is sufficient for this caravan problem.

A Kalman filter operates in two steps. 

\subsection{A Priori Measurements}
In the first step, the Kalman filter performs an \textit{a priori} estimate, through which
it produces an estimate of the state at the current discrete time step based
only on the previous time step by forward-simulating the system dynamics:

\begin{align*}
  \bar{x}_{k} &= A_{d} \hat{x}_{k-1} + B_{d} u_{k} \\
  \bar{P}_{k} &= A_{d} \hat{P}_{k-1} A_{d}^{T} + Q_{k}
\end{align*}

Here, the P matrix is the covariance matrix of the system's states, while Q is a
measure of the system's process noise. Discrete-time process noise is modeled
as follows:

\begin{equation*}
  \renewcommand*{\arraystretch}{2}
    Q_{k} = 
  \begin{bmatrix}
    \frac{T^5}{20} & \frac{T^4}{8} & \frac{T^3}{6} \\
    \frac{T^4}{8} & \frac{T^3}{3} & \frac{T^2}{2} \\
    \frac{T^3}{6} & \frac{T^2}{2} & T
  \end{bmatrix}
  \tilde{q}
\end{equation*}

Where T is the period between each measurement, here defined as 0.1s, and
$\tilde{q}$  is the power spectral density (PSD) of noise for each component.
For our system, we used relatively low process noise with a PSD of 1/10 for
position, and 1/20 for velocity. We also note that the values of $A_d$ and $B_d$
are the discrete-time system dynamics, which are covered in
section \ref{sec:discrete_time}.

\subsection{A Posteriori Measurements}
Next, the Kalman filter takes new measurements into the system and incorporates
them. The Kalman filter creates a model for measurements and compares it to the
actual measurement data it receives, and weights it's \textit{a priori} estimate
with the new information to create a new, \textit{a posteriori} measurement. The
\textit{a posteriori} measurement update equation:

\begin{align*}
  \bar{z}_{k} &= H \bar{x}_{k} \\
  P_{zz} &=  H \bar{P} H^{T} + R_{k} \\
  W &= \bar{P} H^{T} P_{zz}^{-1} \\
  \hat{x}_{k} &= \bar{x}_{k} + W (z_{k} - \bar{z}_{k}) \\
  \hat{P}_{k} &= \bar{P} - W P_{zz} W^{T}
\end{align*}

This \textit{a posteriori} formula can take many forms, and the form above is known as the
Joseph's form. The reason it is used in lieu of other forms is that the Joseph's
form equation copes well with sparse transition matrices that may lead to low
precision inversion of the $P_{zz}$ term. As a result, the Joseph's form works
best for this system, which has some very sparse matrices.

The final result, $\hat{x}$, is the most accurate version of the system states
based on both a model of how these states grow, and measurements that relate to
those states.

\subsection{Different Measurements at Different Times}
For our system, two measurements were taken: position of the first vehicle, and
range between each vehicle in the caravan. These measurements arrived at
different time (1 Hz \& 10 Hz), but the Kalman filter described above assumes
that both measurements arrive simultaneously, however this does not occur in our
system. We wish to combine both measurements into the Kalman filter as soon as
these data arrive.

There are two ways to accommodate measurements taken at different times in a
Kalman filter: 
\begin{enumerate}
  \item Set the covariance of missing measurements infinity
  \item Change the size of the matrix $H_k$ at each step to correspond only with
    the present measurements.
\end{enumerate}

This latter method was chosen for this Kalman filter. Changing $H_k$ was not
difficult, while there were numerical issues with an infinite covariance.

\subsection{Simulation}
The Kalman filter was run both with and without control input to ensure that it
was working. In both cases, the filter's error between estimated state and true
state was driven to zero within two minutes. A plot of the estimation error and
variances of estimated state positions is given below. 

\begin{figure}[H]
	\includegraphics[width=\linewidth]{estimation_error_over_time.jpg}
	\caption{Estimation Error}
	\label{fig:Est. Error}
\end{figure}

\begin{figure}[H]
	\includegraphics[width=\linewidth]{estimation_variance_over_time.jpg}
	\caption{Estimated State Variances}
	\label{fig:Est. State Vars}
\end{figure}

As can be seen, the estimation error is driven to zero quickly, and remains
relatively stable throughout the system. In a few rare instances, a noisy
measurement of vehicle position bounced the error up for a second or two, but
these errors were quickly accommodated by the Kalman filter.

Note the increasing variance of state error from trucks one to three. The Kalman
filter's estimate decreased in accuracy along the extent. This makes sense,
given that only the first vehicle has direct position measurements, and
subsequent position estimates must be derived from the combination of that
measurement with other additional noisy measurements. The noisy sensor
measurements cascade down the caravan, decreasing the accuracy of the estimate
along the way.


%%% ----------------------------------------------------------------------
\section{Control System Description}
\subsection{Error State System}
Controllers drive systems to the origin. The nominal system, however, should not
be driven to the origin. Instead an error-state system should be specified where
the error is the deviation of the nominal system from a reference signal.

First, define the reference signal. From the problem statement, the goal is to
maintain a specified distance between the trucks and to keep that at a constant
velocity.
\begin{equation}
  \vec{x_{r}} = 
  \begin{bmatrix}
    x_{1_{r}}         \\
    \Delta x_{12_{r}} \\
    \Delta x_{23_{r}} \\
    v_{1_{r}}         \\
    v_{2_{r}}         \\
    v_{3_{r}}
  \end{bmatrix}
\end{equation}

The error signal is the difference between the nominal and the reference states.
Note that not all states have a corresponding reference signal.
\begin{equation}
  \vec{x_{e}} = 
  \begin{bmatrix}
    x_{1} - x_{2} - \Delta x_{12_{r}} \\
    x_{2} - x_{3} - \Delta x_{23_{r}} \\
    v_{1} - v_{1_{r}}
  \end{bmatrix}
\end{equation}

The error signal dynamics don't conform to the standard $\dot{x} = Ax + Bu$ form
--- there is an affine transform between the error dynamics and the nominal and
reference signals. However, the standard form can be composed via the following
trick: append the reference signal to the nominal state and define zero dynamics
and full observability for the reference substate.

Stack the nominal and reference states into the \textit{stacked state}.
\begin{equation}
  \vec{x_{s}} = 
  \begin{bmatrix}
    x_{1}             \\
    x_{2}             \\
    x_{3}             \\
    v_{1}             \\
    v_{2}             \\
    v_{3}             \\ 
    \Delta x_{12_{r}} \\
    \Delta x_{23_{r}} \\
    v_{1_{r}}
  \end{bmatrix}
\end{equation}

Define the stacked state dynamics.
\begin{equation}
  \dot{\vec{x}}_{s} = 
  \begin{bmatrix}
    v_{1}             \\
    v_{2}             \\
    v_{3}             \\
    a_{1}             \\
    a_{2}             \\
    a_{3}             \\ 
    0                 \\
    0                 \\
    0
  \end{bmatrix}
  =
  \underbrace{
  \begin{bmatrix}
    0 & 0 & 0 & 1 & 0 & 0 & 0 & 0 & 0 & 0 & 0 \\
    0 & 0 & 0 & 0 & 1 & 0 & 0 & 0 & 0 & 0 & 0 \\
    0 & 0 & 0 & 0 & 0 & 1 & 0 & 0 & 0 & 0 & 0 \\
    0 & 0 & 0 & 0 & 0 & 0 & 0 & 0 & 0 & 0 & 0 \\
    0 & 0 & 0 & 0 & 0 & 0 & 0 & 0 & 0 & 0 & 0 \\
    0 & 0 & 0 & 0 & 0 & 0 & 0 & 0 & 0 & 0 & 0 \\
    0 & 0 & 0 & 0 & 0 & 0 & 0 & 0 & 0 & 0 & 0 \\
    0 & 0 & 0 & 0 & 0 & 0 & 0 & 0 & 0 & 0 & 0 \\
    0 & 0 & 0 & 0 & 0 & 0 & 0 & 0 & 0 & 0 & 0
  \end{bmatrix}
  }_{A_{s}}
  \vec{x}_{s}
  +
  \underbrace{
  \begin{bmatrix}
    0 & 0 & 0 \\
    0 & 0 & 0 \\
    0 & 0 & 0 \\
    1 & 0 & 0 \\
    0 & 1 & 0 \\
    0 & 0 & 1 \\
    0 & 0 & 0 \\
    0 & 0 & 0 \\
    0 & 0 & 0
  \end{bmatrix}
  }_{B_{s}}
  \vec{u}
\end{equation}

Define the stacked state observer equation. Note that the reference states are
specified and fully known so there is full state feedback for these states.
\begin{equation}
  \vec{y} = 
  \begin{bmatrix}
    x_{1}             \\
    \Delta x_{12}     \\
    \Delta x_{23}     \\
    a_{1}             \\
    a_{2}             \\
    a_{3}             \\
    \Delta x_{12_{r}} \\
    \Delta x_{23_{r}} \\
    v_{1_{r}}
  \end{bmatrix}
  =
  \underbrace{
  \begin{bmatrix}
    1 & 0 & 0 & 0 & 0 & 0 & 0 & 0 & 0 & 0 & 0 & 0 \\
    1 & -1 & 0 & 0 & 0 & 0 & 0 & 0 & 0 & 0 & 0 & 0 \\
    0 & 1 & -1 & 0 & 0 & 0 & 0 & 0 & 0 & 0 & 0 & 0 \\
    0 & 0 & 0 & 0 & 0 & 0 & 0 & 0 & 0 & 0 & 0 & 0 \\
    0 & 0 & 0 & 0 & 0 & 0 & 0 & 0 & 0 & 0 & 0 & 0 \\
    0 & 0 & 0 & 0 & 0 & 0 & 0 & 0 & 0 & 0 & 0 & 0 \\
    0 & 0 & 0 & 0 & 0 & 0 & 0 & 0 & 0 & 1 & 0 & 0 \\
    0 & 0 & 0 & 0 & 0 & 0 & 0 & 0 & 0 & 0 & 1 & 0 \\
    0 & 0 & 0 & 0 & 0 & 0 & 0 & 0 & 0 & 0 & 0 & 1 \\
  \end{bmatrix}
  }_{C_{s}}
  \vec{x}_{s}
  +
  \underbrace{
  \begin{bmatrix}
    0 & 0 & 0 \\
    0 & 0 & 0 \\
    0 & 0 & 0 \\
    1 & 0 & 0 \\
    0 & 1 & 0 \\
    0 & 0 & 1 \\
    0 & 0 & 0 \\
    0 & 0 & 0 \\
    0 & 0 & 0
  \end{bmatrix}
  }_{D_{s}}
  \vec{u}
\end{equation}

Now define the error state by linearly transforming the stacked state via a
\textit{similiarity transform}.

\begin{equation*}
  \vec{x}_{e} = 
  \begin{bmatrix}
    x_{1}               \\
    \Delta x_{12_{e}}   \\
    \Delta x_{23_{e}}   \\
    v_{1_{e}}           \\
    v_{2}               \\
    v_{3}               \\ 
    \Delta x_{12_{r}}   \\
    \Delta x_{23_{r}}   \\
    v_{1_{r}}
  \end{bmatrix}
  =
  \underbrace{
  \begin{bmatrix}
    1 & 0 & 0 & 0 & 0 & 0 & 0 & 0 & 0 \\
    1 & -1 & 0 & 0 & 0 & 0 & -1 & 0 & 0 \\
    0 & 1 & -1 & 0 & 0 & 0 & 0 & -1 & 0 \\
    0 & 0 & 0 & 1 & 0 & 0 & 0 & 0 & -1 \\
    0 & 0 & 0 & 0 & 1 & 0 & 0 & 0 & 0 \\
    0 & 0 & 0 & 0 & 0 & 1 & 0 & 0 & 0 \\
    0 & 0 & 0 & 0 & 0 & 0 & 1 & 0 & 0 \\
    0 & 0 & 0 & 0 & 0 & 0 & 0 & 1 & 0 \\
    0 & 0 & 0 & 0 & 0 & 0 & 0 & 0 & 1
  \end{bmatrix}
  }_{T}
  \begin{bmatrix}
    x_{1}             \\
    x_{2}             \\
    x_{3}             \\
    v_{1}             \\
    v_{2}             \\
    v_{3}             \\ 
    \Delta x_{12_{r}} \\
    \Delta x_{23_{r}} \\
    v_{1_{r}}
  \end{bmatrix}
\end{equation*}

Now, leverage the similarity transform to define the error state dynamics.
\begin{align*}
  \dot{\vec{x}}_{e} &= T A_{s} T^{-1} \vec{x}_{e} + T B_{s} \vec{u} \\
  y &= C T^{-1} \vec{x}_{e} + D \vec{u}
\end{align*}

Finally, clean up the equation by redefining the system.
\begin{align*}
  \vec{x} &\coloneqq \vec{x}_{e}    \\
  A       &\coloneqq T A_{s} T^{-1} \\
  B       &\coloneqq T B_{s}        \\
  C       &\coloneqq C_{s} T^{-1}   \\
  D       &\coloneqq D
\end{align*}

Now the system is in the standard form.
\begin{align*}
  \dot{\vec{x}} &= A \vec{x} + B \vec{u} \\
  y &= C \vec{x} + D \vec{u}
\end{align*}




\subsection{Discrete-Time, Finite Horizon LQR Controller} \label{sec:LQR}
A discrete-time, finite horizon LQR controller was chosen to bring the trucks
into `caravan' formation. Given a transient state weighting matrix, $Q$, a final
state weighting matrix, $Q_{f}$, and an input weighting matrix, $R$, LQR
controllers minimize the cost function:
\begin{equation*}
  J = \vec{x}_{f}^{T} Q_{f} \vec{x}_{f} + \sum_{k=0}^{N} \left[ \vec{x}_{k}^{T} Q
  \vec{x}_{k} + \vec{u}_{k}^{T} R \vec{u}_{k} \right]
\end{equation*}

The control input at every time, $u_k$, is determined via dynamic programming. A
proof of optimality is not provided in this report but is outlined in
\cite{LQRStanford}. $u_k$ is determined as follows: 
\begin{enumerate}
  \item Set $P_{N} = Q_{f}$
  \item For $t=N, \hdots, 1$: $P_{t-1} = Q + A^{T}P_{t}A - A^{T} P_{t}
    B(R + B^{T}P_{t}B)^{-1}B^{T}P_{t}A$
  \item For $t=0, \hdots, N-1$: $K_{t} = -(R + B^{T} P_{t+1}
    B)^{-1}B^{T}P_{t+1}A$
  \item For $t=0, \hdots, N-1$: $u_{t}=K_{t}x_{t}$
\end{enumerate}

At the beginning of the horizon, the entire LQR control trajectory is
pre-computed and the respective gain matrixes cached. For the duration of the
control period, the optimal control input is the product of the time-respective
gain matrix and the current state.


%%% ----------------------------------------------------------------------
\subsection{LQR Simulation}
The LQG controller was implemented in Matlab. The goal of the controller is to
bring the trucks into the 'caravan' formation within 10 minutes. The trucks
should follow each other at 30 meters/second with a separation of only 5 meters.

\begin{equation*}
 \vec{x}_{0} = 
  \begin{bmatrix}
    x_{1} \\
    x_{2} \\
    x_{3} \\
    v_{1} \\
    v_{2} \\
    v_{3}
  \end{bmatrix} 
  =
  \begin{bmatrix}
    200 \\
    75 \\
    0 \\
    30 \\
    27 \\
    25
  \end{bmatrix} 
\end{equation*}

\begin{equation*}
  \vec{x}_{r} = 
  \begin{bmatrix}
    \Delta x_{12} \\ 
    \Delta x_{23} \\ 
    v_{1}
  \end{bmatrix} 
  =
  \begin{bmatrix}
    5 \\
    5 \\
    30
  \end{bmatrix} 
\end{equation*}

\begin{figure}[H]
	\includegraphics[width=\linewidth]{states_over_time_with_truth.jpg}
	\caption{States Over Time with Full State Feedback}
\end{figure}

\begin{figure}[H]
	\includegraphics[width=\linewidth]{tracking_error_over_time_with_truth.jpg}
	\caption{Tracking Error Over Time with Full State Feedback}
\end{figure}

The LQR controller is optimal for choices of $Q_f$, $Q_k$, and $R_k$, but it
does not guarantee or follow any input or state constraints. This may cause
problems for real-life systems. For example, in the simulation above, the first
truck abruptly brakes and drops from 30 m/s to 22 m/s almost immediately. Not
only is this physically impossible, but it would be very unsafe on a freeway!
Moreover, from the error graph, one can see that the second truck nearly rams
the first truck (error of -5m implies a crash). This LQR controller was
hand-tuned to prevent that from happening, but there are no hard mathematical
constraints preventing that from happening in a real system.

LQR controllers can admit state and input constraints, but the controllers
become more complicated. For this project, no constraints were implemented.


%%% ----------------------------------------------------------------------
\subsection{LQG Simulation}
Together, the Kalman Filter described in section \ref{sec:Kalman_filter} and the
LQR controller described in section \ref{sec:LQR} form a Linear-Quadratic-Gaussian
controller. The state estimate from the Kalman filter replaces the true state,
but optimality is guaranteed via the Separation Principle.

\begin{figure}[H]
	\includegraphics[width=\linewidth]{states_over_time.jpg}
	\caption{States Over Time with Estimated State}
\end{figure}

\begin{figure}[H]
	\includegraphics[width=\linewidth]{tracking_error_over_time.jpg}
	\caption{Tracking Error with Estimated State}
\end{figure}

Clearly the system is tracking and the trucks enter the caravan formation.
However, the tracking performance is noticeably worse than the LQR controller.
The LQR controller was designed assuming that full state would be known, but in
fact, only a noise estimate is available so the performance is worse. 

The LQG controller suffers from the same lack of state and input constraints as
the LQR controller, but worse! The trucks nearly ram each other and have a
rather noise steady-state error. Moreover, there is no guarantee that the LQG
system has any safety margins \cite{doyle1978guaranteed}. Few passengers would
want to ride in these trucks!

For a safer autonomous truck system, both state and input constraints would have
to be implemented. Moreover, the state of the system may be expanded to include
acceleration/jerk estimates and constraints on these states may also be required
to ensure passenger comfort. More sensors and rigorous testing/simulation may be
required to reduce state uncertainty and increase robustness against sensor
noise/outages.


%%% ----------------------------------------------------------------------
\section{Museum Exhibit}
The following section describes a potential museum exhibit aimed at high-school
and middle-school children.

Note that this section is an abridged version of the original project proposal.
In discussion with Dr. Tanaka, we agreed that the primary portion of the project
would be the design, implementation, and simulation of an LQG controller and
that, to make grading easier, we would try to conform to the original project
proposal.

\subsection{Description}
We propose a museum exhibit composed of toy trains on tracks where visitors
attempt to tune an LQR controller to bring three independent locomotives into
the caravan formation. On a 3 meter by 3 meter tables, toy train tracks are laid
out in a wide circle \footnote {The hope is that a wide circle will
approximately represent a long, straight, endless track}. Three battery-powered
locomotives may be placed anywhere on the tracks. Visitors may turn 6 knobs
corresponding to the 6 diagonal entries of the LQR weighting matrices to tune
the system.


\subsection{Implementation}
Visitors may walk around all sides of the table. On one side of the table are
six knobs that visitors may twist to tune the LQR controller. On the far end of
the table opposite the knobs is a TV. The TV first shows a short 30-second video
describing the goal of the exhibit, what a 'good' controller looks like, and
what each of the 6 knobs do. Visitors are then free to twist the knobs (a
digital value is displayed on the screen), and then press the 'go' button.

Behind the scenes, a Raspberry Pi B+ (RPB) controls the three locomotives. The
locomotives have been 'hacked' and carry a Raspberry Pi Zero (RPZ) that
modulates the supplied battery voltage to each of the locomotives. By modulating
the battery voltage, the RPZ can directly control the speed of each of the
locomotives. 

A camera is suspended above the table and connected to the RPB. The RPB uses an
image processing algorithm on the incoming camera data to track the location of
each of the locomotives and generate the position and position difference
measurements. The RPB fuses the camera measurements in a Kalman filter
described in section \ref{sec:Kalman_filter} to generate position and velocity
estimates.

After the 'go' button is pressed, the RPB reads in the current analog knob
values and solves for optimal control input values. Once solved, the
RPB streams control input commands to the RPZs via wifi. The RPZs convert the
control input into motor voltages and drive the locomotives. 

If the camera detects that two locomotives have touched (i.e. a crash has
occurred), the trains are stopped, and the visitor is notified of the crash and
given another chance.


\subsection{Price Breakdown}
\begin{center}
  \begin{tabular} { || l c || }
    \hline
    Item & Total Price \\ [0.5ex] 
    \hline\hline
    2 Wooden Train Track          & \$100 \\
    3 Battery-Powered Locomotives & \$150 \\
    1 Raspberry Pi B+             & \$40  \\
    1 Wifi Dongle                 & \$15  \\
    1 Raspberry Pi Camera Module  & \$25  \\
    3 Raspberry Pi Zero W         & \$30  \\
    3 PWM Motor Shields           & \$120  \\
    1 LCD TV                      & \$300 \\ 
    6 Knobs and 1 button          & \$50  \\
    \hline
    Total Price                   & \$830 \\
    \hline
  \end{tabular}
\end{center}

\section{Work Contribution}
Both Connor Brashar and Tucker Haydon contributed equally to this project.


%%% ----------------------------------------------------------------------
\bibliographystyle{ieeetr}
\bibliography{report.bib}
\end{document}


